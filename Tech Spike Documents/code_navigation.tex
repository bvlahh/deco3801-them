\documentclass[10pt]{article}

\usepackage{amsmath}
\usepackage{amssymb,amsfonts}
\usepackage{graphicx}

\pagestyle{plain}

\setlength{\textwidth}{6.3truein}
\setlength{\textheight}{8.7truein}
\setlength{\oddsidemargin}{1.0mm}
\setlength{\evensidemargin}{1.0mm}
\setlength{\topmargin}{-20.5truemm}
\setlength{\parindent}{0.0truemm}
\parskip=2mm

\usepackage{listings}
\usepackage{color}

\definecolor{lightgray}{gray}{0.7}
\definecolor{gray}{rgb}{0.5,0.5,0.5}

\lstset{frame=tb,
  language=C,
  aboveskip=3mm,
  belowskip=3mm,
  showstringspaces=false,
  columns=flexible,
  basicstyle={\small\ttfamily},
  numbers=none,
  numberstyle=\tiny\color{gray},
  commentstyle=\color{lightgray},
  breaklines=true,
  breakatwhitespace=true,
  tabsize=3
}

\title{\bf \large DECO3801 - Tech Spike Code Navigation Document}
\author{\normalsize THEM - Typed HTML5 Evaluation Machine \\ \normalsize Carl Hattenfels, Scott Heiner, Shen Yong Lau, Robert Meyer, Brendan Miller, David Uebergang}
\date{}

\begin{document}

\maketitle

There are two primary components to our project: the website and the parser. The PHP files for the website can be found in the \verb'nginx\html' folder. All of the website files have been coded from scratch. The Python files for the parser can be found in \verb'parser'. Since we are adapting a pre-existing library, \verb'html5lib' for this parser, some files relate to the library and some to our application. The html5-python folder (\verb'parser\html5-python') contains all of the code of html5lib, the main parsing library behind our site. We have made a number of additions and modifications to this code base, all of which have been marked with a ``DECO3801" tag to signify where we have made these changes. The following directory listings comprise all of the files that have been modified.

\begin{verbatim}/html5-python/html5-lib/html5parser.py
/html5-python/html5-lib/constants.py
/html5-python/html5-lib/inputstream.py
\end{verbatim}

The source code found in the ``src" directory contains the source code for the concurrent JSON-RPC folder. The file {\verb rpc_server.py} is the current version in use.
\end{document}