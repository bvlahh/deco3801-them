\documentclass[10pt]{article}

\usepackage{amsmath}
\usepackage{amssymb}
\usepackage{graphicx}

\pagestyle{plain}

\setlength{\textwidth}{6.5truein}
\setlength{\textheight}{8.7truein}
\setlength{\oddsidemargin}{2.0mm}
\setlength{\evensidemargin}{2.0mm}
\setlength{\marginparwidth}{20pt}
\setlength{\topmargin}{-24.5truemm}
\setlength{\parindent}{0.0truemm}
\parskip=0.5mm

\title{\bf DECO3801 Critical Evaluation and Reflection}
\author{\normalsize THEM - Typed HTML5 Evaluation Machine \\ \normalsize Carl Hattenfels, Scott Heiner, Shen Yong Lau, Robert Meyer, Brendan Miller, David Uebergang}
\date{}

\begin{document}

\maketitle

\section*{What did we learn from this project?}

One of the primary things we learned in this project was the HTML5 specification! Beyond that, much of what the other members of the group brought to the table was new to the other members. Some had not completed any web development since leaving high school, and working on a live server was a new experience for most of the team. The strategies we developed are highlighted more in the next section, but we believe we learnt a lot about managing a group with varying levels of skill this semester. The pair programming concept for the website side and the parser side was something new we tried as well. All of the design concepts, including the User Experience Goals, portion of the course was completely new to most of the group, and definitely something we had not considered before. What experience was the website going to bring to users? This was something we had to quantify to better understand how the website worked.

As previously mentioned, this project, for most of the group, was their first experience with dealing with a remotely hosted website, outside of the typical working zones used for other course based web development or locally hosted systems. Many of the concepts involved in interfacing the PHP side of the project to the Python parser were new to us and proved to be interesting to work with. We had only very brief interaction with JSON during completion of other courses some of the group had taken, like INFS3202, and we have a new-found appreciation of this information passing system after being forced to use it in this project via JSON-RPC as a means of communication between the PHP and Python sides of the project. Sometimes, members of the team felt like they were the weak link in the development side of the project and at times could not keep up with conversation about the project as well as they would have liked. We tried to do what we could individually on the parser, so we were fairly confident each individual working on the project code itself at least did their part for the overall completion of the project. The project allowed Robert and Carl, who worked on the back-end, to revisit Python, which Robert had not touched upon since first year, as well as getting to know the HTML5 specification, along with a few web development based concepts neither of them had previously come across.

User testing proved to be an interesting experience. Early in the project we had intended to regular user testing on a fortnightly cycle. This ultimately never occurred, with the first round of user testing being undertaken during the test plan assessment. Due to an unfortunate error in the email asking for former DECO1400 students in the Studio 1 subject, DECO1800, we received no response from potential test subjects. Ultimately we relied on friends to fill the role. The response focused mostly on the design of the site, rather than targeted user testing focused on the content of the error messages - a focus of the site which we thought really needed attention from an early stage. Unfortunately, we had not implemented enough of the error checking to fully test the error messages anyway but there was never a follow up phase of user testing to account for this aspect of the site. Since user testing typically lasted only 30 minutes or so, it was impossible to cover all of the errors given the low turnout for testers. In the end it was not a complete waste, as the troubles experienced during user testing highlighted the importance of performing such tests at regular periods during the design and implementation of a project.

Shen personally came from an accounting background and found working on this project to be a really new type of experience for him. As a team, most of us have learnt a plethora of new terms in the programming field, but Shen in particular found he vastly improved his programming vernacular. He had only done a very little HTML 4 coding in his previous semester, and the leap from there to work on this project allowed him to learn several crucial elements of the HTML5 specification. Furthermore, it was an inspiring project to work on, as we have been working with several passionate programmers in the group. Additionally, the knowledge that we gained as a collective from this project has certainly built a good foundation towards each of our own personal understandings on web development and web tools. For instance, in the front end, PHP forms the primary structure of our website and in the back end, the combination of Python and MySQL have been used to handle user queries and database storage. Unfortunately, some team members, in particular Shen and David, felt that their programming skill posed a huge gap in what needed to be achieved in this project and what they could actually accomplish.

\section{How was the team managed? What issues did we encounter?}

From a project manager stand-point, Scott believed he did not end up understanding the team dynamic well enough. He put too much focus in whole group collaboration in person that neglected the team's ability to cohesively work together without person to person interaction more than once a week in the Monday tutorial session. This Monday tutorial session could have been used as a catch up period, instead, where group members would come together and discuss the work completed over the week gone by, and any other pertinent discussions taking place online as necessary. This lead to missed meetings outside of this Monday tutorial session (and sometimes even in this Monday tutorial session), and in turn a lack of cohesion during most sections of the project. Scott had adopted this dynamic from his group formed in the previous semester, which worked on an augmented reality game and relied on the in-person contact to generate ideas for the game and how it would work. Instead, this HTML5 evaluation project did not require anywhere near this level of in-person interaction, and our group's reliance on these meetings constantly brought the team down. Often, the meetings had some users completely left out of the discussion, and other times a great deal of work was determined and distributed, but then spent working on these problems. Meetings of that kind could easily be communicated electronically, leaving the in-person meetings for idea generation and collaboration.

The overall group structure was for the most part, much more thought out compared with other projects group members had dealt with previously at university. The team was structured, as previously mentioned, such that two people would work on the front and backend of the website with another two working on the parser. This formed a very loose form of pair programming, which allowed for a more focused development style, while offering the ability for each group to bounce ideas around. This allowed for more precision on particular aspects of the individual development processes rather than inviting distracted by other project facets and elements.

Two of the members had very weak programming ability, which put a lot of pressure on the rest of the group as far as the development side of the project went. Ideally these two would have handled the documentation, leaving the development team to handle implementation. Unfortunately, most of the documentation and report writing was handled by one of the website developers, which took up a lot of his time. This cascaded into putting a lot more pressure on the second website developer than was necessary, given the team size. This could have been avoided if, earlier one, we allowed some time to introduce ourselves better, and understand our strengths and weaknesses. Then, straight away, everyone would have been on the same page in the project, and we could distribute tasks effectively.

The group has, on the whole, been monitored and arranged in a somewhat effective manner. Undoubtedly, there are several issues that have been emerged during the work progress. Firstly, the two additional members mentioned about that worked on some elements of the documentation tasks and some design work had poor communication between the two of them. For instance, when both of us were compiling scenarios for the test plan document, both members utilised a totally different document file type for each set of scenarios constructed. This has eventually caused our project manager additional difficulty in time spent merging those files. Going beyond this, though, both Shen and David had done digital video production before in previous semesters and believed themselves to be fairly confident with video editing. However, there were limited opportunities for both of them to use my skill here, and in any case, Scott ended up editing the videos himself, as he did not have time to transfer footage between users. Overall, these two members believe their individual contributions to this project have been below what they expected, primarily due to the knowledge gap in terms of programming, but also subsequently due to poor management.

\section*{How well did the design fit the project?}

The design fits the specification very well. It is very understated, providing the information the user needs to fix any errors in their code in a relatively easy to understand manner. Compared to some of the other groups doing the same project, their choice to head towards overly flashy and ultimately Javascript-heavy designs, while very aesthetically pleasing, seemed to get in the way of conveying the necessary information to the user, and losing additional flexibility of the project itself. In this sense, we believe our more simplistic design style is better suited to the intended use of the site.

However, for some of the more technical members of our group, the media and marketing aspect of the assignment was an unfamiliar, and perhaps even slightly unwelcome, aspect of the project. Given the nature of the product, it was difficult for some of us to see the viability and worth of it as a item users could purchase or place investment in its brand. While we could all see the inherent value of the what we had created for personal and academic use, it was not clear how attempts to market our solution to the wider public could be beneficial. It appeared to some of us that what we were creating targeted a very niche market, and a market for which the financial incentive for this product was low.

Since the program attempted to meet a very base-level user and be seen as a short interaction, ``code in, code out" tool, efforts to market it would rely solely on its inherent flexibility as a selling point, to give it the allure of being a jack-of-all trades, or a one-tool-fits-all solution to multiple problems. However, this concept was realised later one, and was not an initial point of design; as such, we could have taken this idea further and made the system much more flexible to handle different parsers better. As it is, changing parsers requires a complete switch of the Python parser to another parser with identical JSON-RPC output, as well as modifications to the error codes and messages. As it stands, the project will remain a HTML5 Evaluator solely, though the selling point of the flexible nature of parser-switching proved a good point of sale.

\section{How well did the final product meet our expectations?}

Most of our expectations were well and truly surpassed by the final implementation. The tool is slick and easy to use, good at informing users of their problems, and was well received by users in the testing phase. It gives users relevant feedback on their website code and structure, and helps to inform them of the methods of solution. Not only this, but the flexibility and dynamic qualities of the system due to our initial idea of website / parser separation gives the tool increased benefits towards its use not just in DECO1400, but other courses relying on quick and easy validation of files, after some minor modifications to the tool. The program can be taken much further than we have, but the groundwork we have laid down has allowed it this flexibility. We have enjoyed seeing the Typed HTML5 Evaluation Machine come into life. Although some aspects of functionality are missing, namely detecting when a CSS or image file is outside of its specific folder, we feel that the product we have delivered is complete and robust.

We were pleasantly surprised with how the project went as a whole. Our team contained a great deal of talent, which made the implementation process far smoother than expected. That is not to say the project could not have exceeded these expectations, as we still believe having at least one more developer with experience with web development would have taken the pressure off of the team focused on implementing the site, and allow our project manager to focus more on documentation and management. The site itself is far better than existing parsers which were used during some of our team members' time taking the very introductory web development subject we were building this for. As such, we have no doubts that our site will prove useful to new students taking DECO1400.

While we were fairly happy with the project at the time of the exhibition, we had made a fatal miscalculation. We had laboured under the misapprehension that, while our product would be partially evaluated at the exhibition, the final assessment of our product would occur after we had submitted all the relevant materials. Unfortunately we realised far too late that this was not the case. Whilst we are required to submit our final product as part of the final submission, the actual evaluation of our program's capability occurred during our exhibition. This had a dual negative effect on our resulting performance in both of these stages. 

We worked hard to get our product to a state we were proud of for the exhibition. However, we neglected some small issues with the understanding that we would have ample time to resolve them prior to final submission of the project, and that they were not an impediment to the usability of our product for the exhibition. Had we realised in advance that our product was being assessed in its entirety at that point then we would have expended more of our resources on development prior to the exhibition, rather scheduling some of our enhancements until afterwards. The second downside was that we expended substantial effort post-exhibit on improving our product for submission. Had we been more vigilant in tracking the requirements of our assessment, we would have been able to allocate our resources more effectively in the post-exhibit period. This was quite a disappointment as many in the group had sacrificed time that would have been better spent on more pressing matters for DECO3801 and other subjects they were undertaking along side it.

Collaborative working yields many clear advantages. However, this entire experience highlights one of the possible pitfalls of teamwork. No-one in our group was specifically delegated the task of checking that we were fulfilling the requirements of our assessment pieces. Though we all took a perfunctory interest in such details at different times throughout the semester, ensuring that we were meeting the requirements well enough, its clear that we failed to keep adequate track of who was keeping track. This is just one example of the propensity for small tasks to fall between the cracks when it comes to delegating responsibility. In future team projects, if we were to work with the members of this group again, we would communicate, and then play to, our individual team members' strengths, rather then stumbling through what was left unshared between us. Members could focus on the sections of the project they felt comfortable with, have time to learn some of the concepts and techniques they were unfamiliar with, and ultimately leave the project feeling fulfilled and well-rounded in terms of the work they undertook as part of it.

\end{document}